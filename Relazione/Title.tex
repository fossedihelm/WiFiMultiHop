\documentclass{llncs}
%
\usepackage{makeidx}  % allows for indexgeneration
\usepackage[utf8]{inputenc}
\usepackage{epstopdf}
\usepackage{float}
\usepackage{courier}
\usepackage{color}
\usepackage[usenames,dvipsnames]{xcolor}
\usepackage[export]{adjustbox}
\usepackage{graphicx}
\usepackage{hyperref}



%
\newcommand{\exedout}{%
	\rule{0.8\textwidth}{0.5\textwidth}%
}

\begin{document}
	\mainmatter              % start of the contributions
	%
	\title{WiFi Direct Multi-Group Hopping}
	%
	\author{Federico Fossemò, Gianluca Guidi}
	%
	\institute{Università di Bologna, Dipartimento di Scienze dell'Informazione
		\email{federico.fossemo@studio.unibo.it, gianluca.guidi3@studio.unibo.it}}
	\maketitle
	
	\begin{abstract}
	\end{abstract}
	%
%%%%%%%%%%%%%%%%%%%%%%%%%%%%%%%%%%%%%%%%%%%%%%%%%%%%%%%%%%%%%%%%%%%%%%%%%%%%%%%%%%%%%

%%%%%%%%%%%%%%%%%%%%%%%%%%%%%%%%%%%%%%%%%%%%%%%%%%%%%%%%%%%%%%%%%%%%%%%%%%%%%%%%%%%%%
\section{Introduzione}
%	\begin{itemize}
%		\item 
%	\end{itemize}

\paragraph{} Questo lavoro, inserito nel contesto del corso di Sistemi Mobili, consiste nello studio di alcune potenzialità di WiFi Direct.\\
WiFi Direct consente la formazione di gruppi di comunicazione peer-to-peer. Ogni gruppo è formato da un GO (Group Owner), che agisce come access point, e più client connessi al GO. I gruppi sono indipendenti tra loro e lo standard non prevede una procedura per connettere più gruppi tra loro, per formare dei cluster più ampi.
\paragraph{} Un possibile modo per collegare più gruppi tra loro consiste nell'avere un client che si connette a più GO, rimanendo connesso a ciascuno di essi per un determinato perdiodo di tempo. Il client, saltando da un GO all'altro ed inoltrando i messaggi ricevuti, è in grado di agire da ponte tra i gruppi, permettendo lo scambio di messaggi tra dispositivi appartenenti a gruppi diversi.
\paragraph{} Lo scopo del progetto presentato è quello di fornire un proof of concept del metodo appena descritto, analizzare i ritardi introdotti ed eventualmente trovare un tempo di permanenza nel gruppo adeguato. L'implementazione consiste in un'applicazione Android che fa uso del framework WifiP2p dello stesso. Lo scenario analizzato è quello in cui sono presenti due Group Owner ed un client se si connette in alternanza ad essi.

%	\begin{enumerate}
%		\item \textit{}
%	\end{enumerate}
	
	

%%%%%%%%%%%%%%%%%%%%%%%%%%%%%%%%%%%%%%%%%%%%%%%%%%%%%%%%%%%%%%%%%%%%%%%%%%%%%%%%%%%%%
\section{Scenario del test}
\paragraph{} I dispositivi Android a disposizione per questo progetto sono 3: essi sono sufficienti per creare le condizioni minime per poter testare l'idea del salto tra gruppi: due dispositivi agiscono come Group Owner, mentre il terzo agirà come client che si connette in alternanza tra i primi due, rimanendo connesso a ciascuno per un periodo di tempo prefissato.
\paragraph{} Ai due Group Owner sono stati assegnati due ruoli. Il primo, detto generator, genera periodicamente messaggi da inoltrare. Il secondo, detto replier, non fa altro che rispedire al mittente i messaggi ricevuti. Il generator prende nota dei tempi di invio e di ricezione, in modo da poter calcolare l'RTT (round trip time) presente tra sè stesso ed il replier. La stima di questo RTT è influenzata dai tempi di processamento e dal tempo di trasmissione dei messaggi. Tuttavia il tempo di processamento dovrebbe essere molto più basso rispetto ad i tempi di ritardo e trasmissione, mentre la piccola dimensione dei messaggi fa sì che la misurazione finale non venga distorta in maniera significativa.
%	\cite{relatedWork}

%	\ref{leo_arch}

%	\begin{figure}[H]
%		\includegraphics[scale=0.3,center]{img/}
%		\caption{}
%		\label{label}
%	\end{figure}
%	\noindent

%%%%%%%%%%%%%%%%%%%%%%%%%%%%%%%%%%%%%%%%%%%%%%%%%%%%%%%%%%%%%%%%%%%%%%%%%%%%%%%%%%%%%		 
\section{Architettura}
\paragraph{} In questa sezione verrà descritta a grandi linee l'implementazione dell'applicazione usata per i test.
		
%%%%%%%%%%%%%%%%%%%%%%%%%%%%%%%%%%%%%%%%%%%%%%%%%%%%%%%%%%%%%%%%%%%%%%%%%%%%%%%%%%%%%
\section{Section4} 


%%%%%%%%%%%%%%%%%%%%%%%%%%%%%%%%%%%%%%%%%%%%%%%%%%%%%%%%%%%%%%%%%%%%%%%%%%%%%%%%%%%%%			
\section{Section5}


%%%%%%%%%%%%%%%%%%%%%%%%%%%%%%%%%%%%%%%%%%%%%%%%%%%%%%%%%%%%%%%%%%%%%%%%%%%%%%%%%%%%%			
\section{Section6}

	
%%%%%%%%%%%%%%%%%%%%%%%%%%%%%%%%%%%%%%%%%%%%%%%%%%%%%%%%%%%%%%%%%%%%%%%%%%%%%%%%%%%%%			
\section{Section7}

	
%%%%%%%%%%%%%%%%%%%%%%%%%%%%%%%%%%%%%%%%%%%%%%%%%%%%%%%%%%%%%%%%%%%%%%%%%%%%%%%%%%%%%			
\section{Section8}


%%%%%%%%%%%%%%%%%%%%%%%%%%%%%%%%%%%%%%%%%%%%%%%%%%%%%%%%%%%%%%%%%%%%%%%%%%%%%%%%%%%%%
\begin{thebibliography}{}
%	\bibitem{}
\end{thebibliography}

\end{document}